\section{Background Research}
Hockey is a one/two season sport, and generally the league season lasts for around half a year. This leaves the players with a lot of off-season time. During this off-season time, the players must keep in shape for the following season, hence why dryland and off-season training is so important. There are two different forms of dryland training, one is off-season and two is in-season \cite{1}. The difference is the off-season training will be when the season is finished and the in-season is when the season has started. Along with the time of these training sessions being different, also comes the training itself being different. During the off season, the foundation of the dryland training is preparing one's body and skills for the following season. To do so, a lot of core strength, stability, balance, and speed will be the main focuses of the training \cite{2}. While during the season (in-season) the workouts will be more about maintaining the player’s strength and fitness, however not at a high enough level where it will exhaust the players. This is because the players will also be working hard in practice and during the game, so they will already maintain a certain level of fitness through doing so. In the off season however, the players will challenge themselves far more, as they do not have the games and as many practices. Overall, this training is very important as if the players are not in top shape come the next season of hockey, then they will not be ready for the season.\par
The group’s project is focusing on assessing the movements associated with hockey and comparing on-ice hockey to dryland hockey training. During the season games and practices, players can only maintain their strength, balance and fitness to a certain extent. So, when they are working off the ice (dryland training), the exercises the players are performing must be properly targeting the muscles that the players need during the game and mimicking the movements that the players are using during their games as well \cite{1}. This comes in many different forms of workouts including off ice rollerblade hockey and many different exercises in the gym. The same applies to off season training, the proper muscles must be targeted, and the proper movements must also be targeted, to be able to exploit weaknesses where the player needs to work, preventing injuries as those muscles will be strong and will have been used and strengthened all off season and overall improving performance \cite{2}.\par
There are multiple ways of evaluating the effectiveness of the training with sensors. Arduino circuit boards are compatible with a variety of sensors and programming libraries that can aid in data collection. The main Arduino board is an Arduino uno rev 2. This board has many built in useful features like an accelerometer, gyroscope, and Wi-Fi connection\cite{3}. The accelerometer and gyroscope will be important for testing as they can take readings on the movements of the board. These  readings will include the direction of moment and changes in the board position. The gyroscope and accelerometer readings will be used to compare the speed and direction that a specific body part is moving. The board's Wi-Fi capabilities allow wireless upload of the data to a database. One of the sensors is an electromyography sensor or EMG. This sensor measures small electric signals created in muscles when they are moved or contracted\cite{5}. Using these sensors on targeted muscle groups, their activation over different exercises can be compared.  In addition to the EMG sensor, there are sets of pressure sensors in the skates.  Using two smaller sensors for the front of the foot and a large one for the heel a player's weight distribution can be evaluated and compared.\par
Hockey is a full body workout, but the focus will primarily be looking at the lower body for skating. To pick the specific muscles that will be compared, it is first needed to test the EMG sensors to see what muscles provide the most useful data. In the initial testing the targets are going to be the calf, quadricep, hamstring and glute muscle groups[6]. In the back of the leg, the calf has the soleus and the gastrocnemius. These muscles are the main plantar flexor of the ankle. They allow one to push the foot out. On the shin there is the tibialis anterior muscle. This is the strongest dorsiflexor for the foot and is responsible for pulling the foot in. On the front of the thigh there are four quadricep muscles, the rectus femoris, vastus intermedius, vastus lateralis and vastus medialis. The rectus femoris is a muscle that passes from the knee to the hip acting as a hip flexor and to extend the knee. The vastus intermedius is covered by the rectus femoris so an EMG will be unable to read it.  The vastus lateralis and vastus medialis are located on either side of the vastus intermedius and rectus femoris. The lateralis is the exterior and the medialis is the interior. These muscles also work together to extend the knee. In the back of the thigh there are the three hamstring muscles the biceps femoris, semitendinosus and semimembranosus. These muscles work to flex the knee joint. The last section is the glutes. They contain the gluteus medius, gluteus minimus, gluteus maximus and the lateral rotators. The gluteus minimus and the lateral rotators are located under the gluteus maximus so they cannot be read by an EMG. The gluteus maximus controls the external rotation of the thigh at the hip. The gluteus medius is responsible for the abduction of the hip joint.\cite{7} 
