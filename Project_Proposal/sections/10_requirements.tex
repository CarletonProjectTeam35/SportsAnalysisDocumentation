\section{Requirements}
\subsection{Hardware Requirements}
\subsubsection{Arduino}
The Arduino Uno Microcontroller unit will be used to read inputs from a vast array of hardware components and sensors and convert it to output that will be used by the software end to analyze and display data in real-time. The arduino unit along with the corresponding hardware and sensors will be mounted to the test subject and wirelessly transmitting data to the database. 

\begin{figure}[htbp]
\centering
\includegraphics[width=0.5\textwidth]{Project_Proposal/figs/arduino_uno.png}
\caption{The Arduino Uno Microcontroller Unit \cite{3}}
\label{fig:arduino}
\end{figure}
\subsubsection{EMG Sensors}
Electromyography (EMG) sensors measure electrical signals generated by a person’s muscle movements. These sensors will be used to collect data on a subject’s muscle movements during the execution of a certain movement.
\begin{figure}[htbp]
\centering
\includegraphics[width=0.5\textwidth]{Project_Proposal/figs/MFG_SEN-13723.png}
\caption{The Electromyography Sensor \cite{8}}
\label{fig:emg}
\end{figure}
\subsubsection{Force Sensitive Resistors}
Force sensitive resistors are sensors that measure applied force or pressure. These sensors will be used on the bottom of the subject’s foot to measure the amount of force being applied during certain movements.
\subsubsection{Accelerometer}
Accelerometers are sensors that measure proper acceleration. These will be used to measure a subject’s acceleration, which will be integrated over time to obtain the speed of a subject. 
\begin{figure}[htbp]
\centering
\includegraphics[width=0.3\textwidth]{Project_Proposal/figs/accelerometer.jpg}
\caption{The Accelerometer Gyroscope Sensor \cite{9}}
\label{fig:gyro}
\end{figure}
\newline
\subsubsection{PicoScope}
A PicoScope is a device that measures real-time signals and operates as an advanced oscilloscope and spectrum analyzer. The PicoScope will be used to measure the output of each of the sensors, allowing the data to be visually observed as a waveform.
\begin{figure}[htbp]
\centering
\includegraphics[width=0.3\textwidth]{Project_Proposal/figs/PicoScope.png}
\caption{The PicoScope\cite{10}}
\label{fig:Pico}
\end{figure}
\subsection{Software Requirements}
For the project’s embedded development, Arduino’s integrated development environment (IDE) will be used, in conjunction with Arduino’s programming language which is based on the C programming language. This will be used to write all the embedded code that will control the hardware, collect the data and subsequently send it to the database. 

Once the data is collected by the hardware, the arduino will send this data wirelessly to a MySQL database using phpmyadmin. MySQL is an open-source relational database that will permit high-speed fetching of data using SQL queries such that the data can be displayed in real-time. The python application will be able to access the MySQL database to acquire the data.

In case implementing the MySQL database and Arduino connection proves to be too difficult or inefficient, the plan will pivot to using Matlab’s ThingSpeak software for data storage/management, and analytics. ThingSpeak is an Internet of Things API that allows data to be collected, stored, analyzed and graphically represented in the cloud, in real-time \cite{11}. Using Thingspeak’s Arduino API, the Arduino board will be able to seamlessly send data to Thingspeak.

On the front-end, a cross-platform Python application running on a local machine will pull down the collected data from the database in real time, process and analyze the data, using a variety of Python mathematical, data analytics and machine learning libraries, and present it on a user interface dashboard. This dashboard will allow the team to monitor select data and results as they are collected in real-time, and allow for adjustments to be made as needed. Python is an adept tool for these tasks as it is widely used in the data science world due to its abundance of library offerings such as NumPy, Pandas, Matplotlib that make data cleaning, analysis and visualization much more efficient and simple \cite{11}. Further, Python offers many frameworks for building graphical user interfaces which allows the team to maintain a consistent language for all front-end needs, while maintaining cross-platform capabilities. As python is such a widely used language on the data science and UI front, there is an abundance of documentation and information online that will help the team in successfully developing and testing the application.
\begin{figure}[htbp]
\centering
\includegraphics[width=\textwidth]{Project_Proposal/figs/CapstoneDiagrams.png}
\caption{Diagram of the Software Process}
\label{fig:Software}
\end{figure}
\subsection{Facility Requirements}
In order to accomplish data collection and testing for the ice skating portion of the project, an indoor, controlled and maintained ice skating rink will be needed. For this rink the team will be using the Ice House at Carleton University. For the roller blading portion of data collection, an indoor, controlled and maintained gymnasium will be required. The gymnasium that will be used will be Norm Fenn at Carleton University. 
